% (c) 2014 Daniele Masini - d.masini.it@gmail.com
\section{Esercizi}

\subsubsection*{\thechapter.1 - Luoghi geometrici}

\begin{multicols}{2}

\begin{esercizio}
\label{ese:5.1}
Dimostra che il luogo geometrico dei punti del piano equidistanti da due rette incidenti (con il punto $P$ in comune) è l'unione delle due rette, perpendicolari tra loro, che costituiscono le quattro bisettrici degli angoli (di vertice $P$) individuati dalle due rette.
\end{esercizio}

\begin{esercizio}
\label{ese:5.2}
Dimostra che il luogo geometrico dei punti del piano equidistanti da due rette parallele e distinte, $r$ ed $s$, è la retta $t$, parallela ad entrambe, interna alla striscia di piano compresa tra $r$ ed $s$, che divide la striscia in due strisce congruenti.
\end{esercizio}

\begin{esercizio}
\label{ese:5.3}
Dagli estremi $B$ e $C$ della base di un triangolo isoscele $ABC$ condurre le perpendicolari al lato obliquo, più precisamente, per $B$ condurre la perpendicolare ad $AB$, per $C$ la perpendicolare ad $AC$. Detto $D$ il punto in cui si incontrano le due perpendicolari, dimostrare che $AD$ è asse di $BC$.
\end{esercizio}

\begin{esercizio}
\label{ese:5.4}
Nel triangolo $ABC$ con $AB$ maggiore di $AC$, condurre la bisettrice $AD$ dell'angolo in $A$. Dal punto $D$ traccia una retta che incontri $AB$ nel punto $E$ in modo che $A\widehat{D}C\cong A\widehat{D}E$. Dimostra che $AD$ è asse di $CE$.
\end{esercizio}

\end{multicols}
 
\subsubsection*{\thechapter.2 - Circonferenza e cerchio: definizioni e prime proprietà}

\begin{esercizio}
\label{ese:5.5}
Vero o falso?
\begin{enumeratea}
\item Si chiama corda il segmento che unisce il centro della circonferenza a un suo punto\hfill\boxV\quad\boxF
\item Si chiama diametro la corda che passa per il centro\hfill\boxV\quad\boxF
\item Si chiama angolo alla circonferenza un angolo che ha i lati sulla circonferenza\hfill\boxV\quad\boxF
\item Si chiama angolo al centro un angolo che ha per vertice il centro della circonferenza\hfill\boxV\quad\boxF
\item Due corde che si trovano alla stessa distanza dal centro sono congruenti\hfill\boxV\quad\boxF
\item L’angolo alla circonferenza è il doppio del corrispondente angolo al centro\hfill\boxV\quad\boxF
\item Una retta è esterna a una circonferenza se la sua distanza dal centro della circonferenza è maggiore del raggio\hfill\boxV\quad\boxF
\item Due circonferenze che hanno due punti in comune si dicono concentriche\hfill\boxV\quad\boxF
\item Una retta che passa per il centro della circonferenza è sempre secante\hfill\boxV\quad\boxF
\item Una retta tangente a una circonferenza è sempre perpendicolare al raggio che passa per il punto di tangenza\hfill\boxV\quad\boxF
\end{enumeratea}
\end{esercizio}

\begin{multicols}{2}

\begin{esercizio}
\label{ese:5.6}
Dimostra che il luogo dei punti medi delle corde tra loro congruenti di una stessa circonferenza è una circonferenza.
\end{esercizio}

\begin{esercizio}
\label{ese:5.7}
Sia $AB$ il diametro di una circonferenza. Dagli estremi del diametro si conducano due corde $AC$ e $BD$ tra loro parallele. Dimostra che le due corde sono congruenti e che $DC$ è diametro.
\end{esercizio}

\begin{esercizio}
\label{ese:5.8}
Sia $OAB$ un triangolo isoscele. Si tracci la circonferenza con centro in $O$ e raggio $r$ minore di $OA$. Siano $C$ e $D$ i punti di intersezione della circonferenza con i lati obliqui del triangolo isoscele. Dimostra che $ABCD$ è un trapezio isoscele.
\end{esercizio}

\begin{esercizio}
\label{ese:5.9}
Siano $AB$ e $BC$ due corde congruenti di una circonferenza di centro $O$. Dimostra che $AO$ è bisettrice dell'angolo $B\widehat{A}C$.
\end{esercizio}

\begin{esercizio}
\label{ese:5.10}
Sia $AB$ una corda di una circonferenza ed $M$ il suo punto medio. Sia $C$ un punto di $AM$ e $D$ un punto di $MB$ tali che $AC$ sia congruente a $BD$. Condurre da $C$ e da $D$ le perpendicolari alla corda $AB$. Dimostrare che queste perpendicolari incontrandosi con la circonferenza individuano due corde congruenti.
\end{esercizio}

\begin{esercizio}
\label{ese:5.11}
Sia $AB$ una corda di una circonferenza di centro $O$. Si prolunghi $AB$ di un segmento $BC$ congruente al raggio della circonferenza. Dimostrare che l'angolo $A\widehat{O}C$ è il triplo dell'angolo $A\widehat{C}O$.
\end{esercizio}

\begin{esercizio}
\label{ese:5.12}
Siano $AB$ e $AC$ due corde congruenti di una stessa circonferenza. Dimostra che il diametro passante per $A$ è bisettrice dell'angolo alla circonferenza di arco $BC$.
\end{esercizio}

\begin{esercizio}
\label{ese:5.13}
Siano $AB$ e $CD$ due corde congruenti che si intersecano nel punto $E$. Dimostra che il diametro passante per $E$ è bisettrice dell'angolo $AEC$.
\end{esercizio}

\begin{esercizio}
\label{ese:5.14}
Dimostra che se due corde si incontrano nel loro punto medio comune allora necessariamente le corde sono diametri.
\end{esercizio}

\begin{esercizio}
\label{ese:5.15}
Dimostrare che in una circonferenza di diametro $AB$ e centro $O$ il luogo geometrico dei punti medi delle corde con un estremo in $A$ è la circonferenza di diametro $AO$.
\end{esercizio}

\begin{esercizio}
\label{ese:5.16}
In una circonferenza di centro $O$ due corde, $AB$ e $BC$ si incontrano in un punto $P$ interno alla circonferenza tale che $OP$ è bisettrice dell'angolo formato dalle due corde. Dimostra che $AB$ e $CD$ sono congruenti.
\end{esercizio}

\begin{esercizio}
\label{ese:5.17}
Sia $AB$ una corda di una circonferenza di centro $O$ e sia $P$ il punto di intersezione tra la corda e la sua perpendicolare condotta dal centro $O$. Dimostra che ogni altra corda passante per $P$ è maggiore di $AB$.
\end{esercizio}

\begin{esercizio}
\label{ese:5.18}
Sia $AB$ il diametro di una circonferenza e $CD$ una corda perpendicolare ad $AB$. Dimostra che $ACD$ e $BCD$ sono triangoli isosceli.
\end{esercizio}

\begin{esercizio}
\label{ese:5.19}
Dimostra che due corde parallele e congruenti di una stessa circonferenza sono lati del rettangolo che ha per vertici gli estremi delle corde.
\end{esercizio}

\subsubsection*{\thechapter.5 - Proprietà dei segmenti di tangenza}

\begin{esercizio}
\label{ese:5.20}
Partendo dai due segmenti consecutivi e congruenti $OA$ e $AB$ costruire le due circonferenze di centro $O$ e raggio rispettivamente $OA$ e $OB$. Per il punto $A$ si conduca la tangente alla circonferenza di raggio $OA$. Detti $C$ e $D$ i punti in cui la suddetta tangente incontra la circonferenza di raggio $AB$, dimostrare che $OCBD$ è un rombo.
\end{esercizio}

\begin{esercizio}
\label{ese:5.21}
Su una circonferenza di centro $O$ si consideri un punto $C$ e un diametro $AB$; sia $t$ la tangente in $C$ alla circonferenza e siano $A'$ e $B'$ le proiezioni su $t$ rispettivamente di $A$ e di $B$. Dimostrare che $C$ è punto medio di $A'B'$ e che $CO$ è congruente alla semisomma di $AA'$ e $BB'$. 
\end{esercizio}

\begin{esercizio}
\label{ese:5.22}
Una retta $r$ taglia due circonferenze concentriche $C_1$ e $C_2$, siano $A$ e $B$ i punti individuati da $r$ sulla circonferenza $C_1$ e $C$ e $D$ i punti sulla circonferenza $C_2$. Dimostra che $AC$ è congruente a $BD$.
\end{esercizio}

\begin{esercizio}
\label{ese:5.23}
Un triangolo isoscele $ABC$ di base $BC$ è inscritto in un cerchio di raggio $OC$. Prolunga l'altezza $BH$ relativa al lato obliquo $AC$ fino a incontrare la circonferenza in $D$. Quali triangoli rettangoli si ottengono? Quali angoli della figura sono congruenti all'angolo in $D$?
\end{esercizio}

\begin{esercizio}
\label{ese:5.24}
Dimostrare che le tangenti a una circonferenza condotte dagli estremi di un suo diametro sono parallele tra di loro.
\end{esercizio}

\begin{esercizio}
\label{ese:5.25}
Nel triangolo $ABC$ traccia le altezze $AH$ e $BK$. Dimostra che la circonferenza di diametro $AB$ passa per i punti $H$ e $K$.
\end{esercizio}

\begin{esercizio}
\label{ese:5.26}
Date due circonferenze concentriche dimostrare che la corda staccata dalla circonferenza maggiore su una tangente alla circonferenza minore è dimezzata dal punto di tangenza. 
\end{esercizio}

\begin{esercizio}
\label{ese:5.27}
Da un punto $P$ esterno a una circonferenza si conducono le due tangenti alla circonferenza, esse incontrano la circonferenza in $A$ e in $B$. Per un punto $Q$ della circonferenza, diverso da $A$ e da $B$, e dalla parte di $P$, si conduce una tangente alla circonferenza, la quale incontra la tangente $PA$ in $D$ e la tangente $PB$ in $C$. Dimostrare che $A\widehat{O}B\cong 2\cdot D\widehat{O}C$. 
\end{esercizio}

\begin{esercizio}
\label{ese:5.28}
Da un punto $P$ esterno a una circonferenza si conducono le tangenti alla circonferenza che risultano tangenti tra di loro, siano $A$ e $B$ i punti di tangenza. Sia $B$ un punto della circonferenza tale che l'angolo in $A$ è retto. Dimostra che $AC$ è la bisettrice dell'angolo $B\widehat{C}P$.
\end{esercizio}

\begin{esercizio}
\label{ese:5.29}
Dagli estremi del diametro $AB$ di una circonferenza si conducono due corde tra loro congruenti, dimostrare che la congiungente gli altri due estremi delle corde passa per il centro della circonferenza.
\end{esercizio}

\begin{esercizio}
\label{ese:5.30}
Dimostra che unendo gli estremi di due corde parallele ma non congruenti si ottiene un trapezio isoscele.
\end{esercizio}

\begin{esercizio}
\label{ese:5.31}
Sia $AB$ il diametro di una circonferenza, Siano $C$ e $D$ i punti di intersezione di una secante con la circonferenza, $C$ il punto più vicino a $B$ e $D$ il punto più vicino ad $A$. Da $A$ e da $B$ si conducono le perpendicolari alla secante che la intersecano rispettivamente in $H$ e in $K$. Dimostra che $DH$ è congruente a $CK$.
\end{esercizio}

\begin{esercizio}
\label{ese:5.32}
Siano $C$ e $C'$ due circonferenze concentriche, il raggio di $C$ sia doppio del raggio di $C'$. Da un punto $P$ della circonferenza maggiore condurre le due tangenti all'altra circonferenza. Dimostra che il triangolo formato da $P$ e dai punti di tangenza è un triangolo equilatero.
\end{esercizio}

\begin{esercizio}
\label{ese:5.33}
Per un punto $P$ esterno a una circonferenza di centro $O$ traccia le due tangenti alla circonferenza e indica con $A$ e $B$ i due punti di tangenza. Dimsotra che la retta $PO$ è asse di $AB$. Siano $C$ e $D$ i punti di intersezione della retta $OP$ con la circonferenza. Dimostra che i triangoli $ABC$ e $ADB$ sono isosceli. Conduci per $O$ il diamtero parallelo alla corda $AB$, il prolungamento del diametro incontra le tangenti $PA$ e $PB$ rispettivamente in $E$ e in $F$. Dimostra che $PC$ è asse di $EF$. E che $EA$ è congruente a $BF$.
\end{esercizio}

\begin{esercizio}
\label{ese:5.34}
In una circonferenza di diametro $AB$, dagli estremi $A$ e $B$ si conducano due corde parallele $AC$ e $BD$. Dimostra che $AC$ è congruente a $BD$ e che $CD$ è un diametro.
\end{esercizio}

\begin{esercizio}
\label{ese:5.35}
In una circonferenza si disegnino due corde $AB$ e $CD$ congruenti e incidenti in $E$ in modo tale che $AE\cong CE$. Dimostra che gli estremi delle corde sono i vertici di un trapezio isoscele.
\end{esercizio}

\begin{esercizio}
\label{ese:5.36}
In una circonferenza di diametro $AB$ si individuino due punti $D$ e $C$ tali che siano congruenti gli angoli al centro $A\widehat{O}D$ e $A\widehat{O}C$. Dimostra che $BC$ è congruente a $BD$.
\end{esercizio}

\begin{esercizio}
\label{ese:5.37}
Dagli estremi della corda $AB$ di una circonferenza disegna le tangenti alla circonferenza stessa e sia $C$ il loro punto di intersezione. Dimostra che il triangolo $ABC$ è isoscele.
\end{esercizio}

\begin{esercizio}
\label{ese:5.38}
Un triangolo $ABC$ è inscritto in una circonferenza. Disegna l'asse del segmento $AB$ che interseca in $D$ l'arco $AB$ non contenente $C$. Dimostra che $CD$ è bisettrice dell'angolo $A\widehat{C}B$.
\end{esercizio}

\begin{esercizio}
\label{ese:5.39}
Data una circonferenza di centro $O$, da un punto $P$ tale che $PO$ sia congruente al diametro dela circonferenza si conducano le tangenti alla circonferenza e siano $A$ e $B$ i punti di tangenza. Siano $M$ ed $N$ rispettivamente i punti medi di $PA$ e $PB$. Dimostra che i triangoli $ABM$ e $ABN$ sono congruenti.
\end{esercizio}

\begin{esercizio}
\label{ese:5.40}
Siano $t$ e $t'$ due tangenti ad una circonferenza negli estremi di un diametro $AB$. Sul prolungamento del diametro $AB$ dalla parte di $A$ prendi un punto $P$ e da esso conduci una tangente $t''$ alla circonferenza. Siano $R$ ed $S$ i punti in cui $t''$ incontra rispettivamente $t$ e $t'$.  Dimostra che il triangolo $ROS$ è rettangolo in $O$, dove $O$ è il centro della circonferenza.
\end{esercizio}

\end{multicols}

 
\subsubsection*{\thechapter.8 - Proprietà dei quadrilateri inscritti e circoscritti}

\begin{esercizio}
\label{ese:5.41}
Quali dei seguenti gruppi di angoli possono essere angoli interni di un quadrilatero inscritto in una circonferenza?
\begin{enumeratea}
\item $\alpha=80\grado$\tab	$\beta=60\grado$\tab $\gamma=100\grado$\tab $\delta=120\grado$;
\item $\alpha=45\grado$\tab	$\beta=30\grado$\tab $\gamma=45\grado$\tab $\delta=60\grado$;
\item $\alpha=185\grado$\tab $\beta=90\grado$\tab $\gamma=90\grado$\tab $\delta=15\grado$;
\item $\alpha=110\grado$\tab $\beta=120\grado$\tab $\gamma=70\grado$\tab $\delta=60\grado$.
\end{enumeratea}
\end{esercizio}

\begin{esercizio}
\label{ese:5.42}
Quali dei seguenti gruppi possono essere le lunghezze dei lati di un quadrilatero inscritto in una circonferenza?
\begin{enumeratea}
\item $a=80$ cm\tab	$b=60$ cm\tab $c=\np{1000}$ cm\tab $d=120$ cm;
\item $a=\np{4,5}$ cm\tab $b=3$ cm\tab $c=\np{4,5}$ cm\tab $d=3$ cm;
\item $a=\np{18,5}$ cm\tab $b=90$ cm\tab $c=\np{0,5}$ cm\tab $d=100$ cm;
\item $a=110$ cm\tab $b=120$ cm\tab $c=130$ cm\tab $d=120$ cm.
\end{enumeratea}
\end{esercizio}

\begin{esercizio}
\label{ese:5.43}
Di quali delle seguenti figure esiste sempre sia la circonferenza inscritta che quella circoscritta?
\begin{multicols}{2}
\begin{enumeratea}
\item triangolo equilatero\tab\boxV\quad\boxF
\item triangolo isoscele\tab\boxV\quad\boxF
\item triangolo rettangolo\tab\boxV\quad\boxF
\item rettangolo\tab\boxV\quad\boxF
\item rombo\tab\boxV\quad\boxF
\item trapezio isoscele\tab\boxV\quad\boxF
\item quadrato\tab\boxV\quad\boxF
\item parallelogramma\tab\boxV\quad\boxF
\item deltoide\tab\boxV\quad\boxF
\end{enumeratea}
\end{multicols}
\end{esercizio}

\begin{esercizio}
\label{ese:5.44}
Dimostra che in un triangolo la distanza tra l’ortocentro e il baricentro è il doppio della distanza tra baricentro e circocentro.
\end{esercizio}

\begin{esercizio}
\label{ese:5.45}
Il triangolo $ABC$ ha le mediane $BM$ e $NC$ congruenti. Le diagonali si incontrano nel punto $O$. Dimostra che $BON$ è congruente a $COM$.
\end{esercizio}

\begin{esercizio}
\label{ese:5.46}
Dimostra che in un esagono regolare ciascun angolo al vertice è diviso in quattro parti uguali dalle diagonali che partono da quel vertice.
\end{esercizio}

\begin{esercizio}
\label{ese:5.47}
Sia $ABC$ un triangolo equilatero inscritto nella circonferenza di centro $O$, sia $DEF$ il triangolo equilatero simmetrico di $ABC$ rispetto ad $O$. Dimostra che $AFBDCE$ è un esagono regolare.
\end{esercizio}

\begin{esercizio}
\label{ese:5.48}
Sia $ABCDE$ un pentagono regolare; prolunga ciascun lato del pentagono nello stesso verso di un segmento congruente al lato del pentagono. Dimostra che gli estremi dei lati prolungati formano un poligono inscrittibile e circoscrittibile.
\end{esercizio}

\begin{esercizio}
\label{ese:5.49}
Sia $ABCDEF$ un esagono regolare e sia $G$ il punto di intersezione delle diagonali $BE$ e $CF$. Dimostra che $ABGF$ è un rombo.
\end{esercizio}

\begin{esercizio}
\label{ese:5.50}
Sia $P$ il punto di intersezione delle diagonali di un trapezio isoscele. Dimostra che il diametro passante per $P$ della circonferenza circoscritta al trapezio è perpendicolare alle basi del trapezio.
\end{esercizio}

\begin{esercizio}
\label{ese:5.51}
Dimostra che in un triangolo rettangolo, la bisettrice dell'angolo retto è anche bisettrice dell'angolo formato dall'altezza e dalla mediana relative all'ipotenusa.
\end{esercizio}

\begin{esercizio}
\label{ese:5.52}
Dimostra che ogni parallelogramma circoscrivibile a una circonferenza è un rombo.
\end{esercizio}

\begin{esercizio}
\label{ese:5.53}
Una circonferenza di centro $O$ è inscritta in un trapezio, non necessariamente isoscele, di basi $AB$ e $CD$. Dimostra che gli angoli $A\widehat{O}D$ e $B\widehat{O}C$ sono retti.
\end{esercizio}

\begin{esercizio}
\label{ese:5.54}
Dimostra che la circonferenza inscritta e quella circoscritta a un quadrato sono concentriche.
\end{esercizio}

\begin{esercizio}[Olimpiadi della matematica 2005]
\label{ese:5.55}
Sia $ABC$ un triangolo rettangolo in $A$, con $AB > AC$ e sia $AH$ l'altezza relativa all'ipotenusa. Sulla retta $BC$ si prenda $D$ tale che $H$ sia punto medio di $BD$; sia poi $E$ il piede della perpendicolare condotta da $C$ ad $AD$. Dimostrare che $EH = AH$  \emph{Suggerimenti: il triangolo $ABD$ è isoscele su base $BD$ quindi ldots{}; considerare poi la circonferenza di diametro $AC$ a cui appartengono i punti $H$ ed \ldots{}; osservare angoli alla circonferenza \ldots{} archi \ldots{} corde \ldots{}}
\end{esercizio}

\begin{esercizio}[Olimpiadi della matematica 2006]
\label{ese:5.56}
Sia $ABCD$ un quadrilatero; chiamiamo $E$ l'intersezione (distinta da $A$) tra le circonferenze di diametri $AB$ e $AC$ ed $F$ l'intersezione (sempre distinta da $A$) tra le circonferenze di diametri $AC$ e $AD$. Dimostrare che: a) se l'angolo $EAD$ è retto, allora $BC$ è parallelo ad $AD$; b) se gli angoli $E\widehat{A}D$ e $F\widehat{A}B$ sono retti, allora $ABCD$ è un parallelogramma; c) se $ABCD$ è un parallelogramma, allora gli angoli $E\widehat{A}D$ e $F\widehat{A}B$ sono retti. \emph{Suggerimenti: osservare parallelismi e ricordare il teorema di Talete.}
\end{esercizio}

\begin{esercizio}[Olimpiadi di matematica 1998]
\label{ese:5.57}
Dato il triangolo $ABC$ con $C\widehat{A}B-A\widehat{B}C=90\grado$, detti $M$ il punto medio di $AB$ e $H$ il piede dell'altezza relativa ad $AB$, dimostrare che il raggio della circonferenza circoscritta ad $ABC$ è uguale ad $HM$.
\end{esercizio}

\begin{esercizio}[Prove invalsi 2003]
\label{ese:5.58}
Un esagono regolare e un quadrato hanno lo stesso perimetro. Quanto vale il rapporto fra un lato dell'esagono e un lato del quadrato?
\begin{multicols}{2}
\begin{enumeratea}
\item $2/3$;
\item $3/4$;
\item $1$;
\item $3/2$;
\item Dipende dal valore del perimetro.
\end{enumeratea}
\end{multicols}
\end{esercizio}

\begin{esercizio}[Prove invalsi 2005]
\label{ese:5.59}
Osserva la seguente figura. Quale delle seguenti affermazioni relative alla figura è falsa?
\begin{enumeratea}
\item Il triangolo $ABC$ è acutangolo.
\item Il punto $O$ è l'intersezione delle altezze del triangolo $ABC$.
\item Le rette $r$, $s$ e $t$ sono gli assi dei lati del triangolo $ABC$.
\item I punti $A$, $B$ e $C$ sono equidistanti da $O$.
\end{enumeratea}
\end{esercizio}

\begin{esercizio}[Prove invalsi 2007]
\label{ese:5.60}
Osserva la figura. Quale delle seguenti affermazioni è vera?
\begin{enumeratea}
\item Il triangolo è inscritto nella circonferenza minore.
\item Il triangolo è inscritto nella circonferenza maggiore.
\item La circonferenza maggiore è inscritta nel triangolo.
\item Il triangolo è circoscritto alla circonferenza maggiore.
\end{enumeratea}
\end{esercizio}

\begin{esercizio}[Prove invalsi 2007]
\label{ese:5.61}
Osserva la figura. I due angoli $A\widehat{C}B$ e $A\widehat{C'}B$ sono uguali? Quali sono le loro ampiezze in gradi?
\begin{enumeratea}
\item Non sono uguali e $A\widehat{C}B=90\grado$ e $A\widehat{C'}B=60\grado$
\item Non sono uguali e $A\widehat{C}B=60\grado$ e $A\widehat{C'}B=45\grado$
\item Sono uguali e $A\widehat{C}B=A\widehat{C'}B=60\grado$
\item Sono uguali e $A\widehat{C}B=A\widehat{C'}B=90\grado$
\item Sono uguali e $A\widehat{C}B=A\widehat{C'}B=180\grado$
\end{enumeratea}
\end{esercizio}

\begin{esercizio}[Prove invalsi 2003]
\label{ese:5.62}
Nella figura seguente $O$ è il centro della circonferenza, $B$ un punto su di essa e $AC$ un suo diametro. Sapendo che $A\widehat{O}B=80\grado$, quanto vale $C\widehat{A}B-A\widehat{C}B$?
\begin{enumeratea}
\item $5\grado$
\item $10\grado$
\item $15\grado$
\item $20\grado$
\item $40\grado$
\end{enumeratea}
\end{esercizio}

\begin{esercizio}[Prove invalsi 2003]
\label{ese:5.63}
Qual è il massimo numero di punti che una circonferenza e i quattro lati di un quadrato possono avere in comune?
\begin{multicols}{5}
\begin{enumeratea}
\item 2;
\item 4;
\item 6;
\item 8;
\item 10.
\end{enumeratea}
\end{multicols}
\end{esercizio}

\begin{esercizio}[Prove invalsi 2005]
\label{ese:5.64}
Osserva attentamente la figura. Sapendo che $A\widehat{O}B\cong C\widehat{O}D\cong B\widehat{V}C=\alpha$, quanto misura $A\widehat{O}D$?
\begin{multicols}{4}
\begin{enumeratea}
\item $\alpha$;
\item $2\alpha$;
\item $3\alpha$;
\item $4\alpha$.
\end{enumeratea}
\end{multicols}
\end{esercizio}

\begin{esercizio}[Prove invalsi 2005]
\label{ese:5.65}
Qual è il massimo numero possibile di punti di intersezione fra una circonferenza e un triangolo?
\begin{multicols}{4}
\begin{enumeratea}
\item 6;
\item 5;
\item 4;
\item 3;
\end{enumeratea}
\end{multicols}
\end{esercizio}

\begin{esercizio}[Prove invalsi 2005]
\label{ese:5.66}
Quale delle seguenti affermazioni è falsa? 
\begin{enumeratea}
\item In ogni triangolo isoscele l'altezza e la mediana relative alla base e la bisettrice dell'angolo al vertice coincidono.
\item In ogni triangolo isoscele baricentro, incentro, ortocentro e circocentro sono allineati.
\item In ogni triangolo isoscele baricentro, ortocentro, incentro e circocentro coincidono.
\item In ogni triangolo equilatero baricentro, ortocentro, incentro e circocentro coincidono.
\end{enumeratea}
\end{esercizio}

\begin{esercizio}[Prove invalsi 2006]
\label{ese:5.67}
Considera la seguente figura. Se le due circonferenze hanno raggi diversi, quale delle seguenti affermazioni è vera?
\begin{enumeratea}
\item Le due circonferenze sono simmetriche rispetto al punto $O$.
\item Le due circonferenze sono simmetriche rispetto a ciascuna delle rette $r$ e $s$.
\item ??
\item ??
\end{enumeratea}
\end{esercizio}

\begin{esercizio}
\label{ese:5.68}
Nella figura seguente il punto $O$ è il punto medio del diametro $AC$. L'angolo $A\widehat{O}B$ misura $40\grado$. Quanto misura l'angolo $O\widehat{B}C$? 
\begin{multicols}{4}
\begin{enumeratea}
\item $10\grado$;
\item $20\grado$;
\item $40\grado$;
\item $60\grado$.
\end{enumeratea}
\end{multicols}
\end{esercizio}
