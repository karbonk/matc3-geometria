% (c) 2014 Daniele Masini - d.masini.it@gmail.com
\section{Esercizi}

%\begin{multicols}{2}

\begin{esercizio}
\label{ese:8.1}
Le figure delle seguenti coppie si corrispondono in una trasformazione geometrica piana: associate a ciascuna coppia di figure la caratteristica che rimane immutata nella trasformazione, ossia individuate l'invariante o gli invarianti della trasformazione.
\end{esercizio}

\begin{esercizio}
\label{ese:8.2}
Si sa che una trasformazione geometrica muta un quadrato in un rombo; gli invarianti di questa trasformazione sono:
\begin{enumeratea}
\item il parallelismo dei lati e l'ampiezza degli angoli;
\item l'ampiezza degli angoli e la misura dei lati;
\item solo il parallelismo 	dei lati;
\item il parallelismo dei lati e la perpendicolarità delle diagonali.
\end{enumeratea}
\end{esercizio}

\begin{esercizio}
\label{ese:8.3}
Quali coppie sono formate da figure corrispondenti in una isometria?
\end{esercizio}

\begin{esercizio}
\label{ese:8.4}
Presi nel piano due punti $T$ e $T'$ è vero che possiamo sempre individuare la simmetria centrale in cui $T'$ è immagine di $T$?
\end{esercizio}

\begin{esercizio}
\label{ese:8.5}
Come dobbiamo scegliere due segmenti affinché sia possibile determinare una simmetria centrale in cui essi siano corrispondenti?
\end{esercizio}

\begin{esercizio}
\label{ese:8.6}
Anche in natura si presentano elementi dotati di un centro di simmetria: cercate una foto di un fiore che presenta un centro di simmetria e individuate il centro di simmetria.
\end{esercizio}

\begin{esercizio}
\label{ese:8.7}
Sappiamo che $S_K:P\left(\frac{3}{5};0\right) \rightarrow P'\left(-\frac{2}{3};-\frac{1}{2}\right)$, determinate il centro $K$ della simmetria. 
\end{esercizio}

\begin{esercizio}
\label{ese:8.8}
Il segmento di estremi $A(-2;4)$ e $B(2;-4)$ in $S_O$, essendo $O$ l'origine del riferimento cartesiano ortogonale
\begin{enumeratea}
\item ha tutti i suoi punti fissi;
\item ha un solo punto fisso;
\item ha fissi solo gli estremi;
\item ha fissi tutti i punti interni ma non gli estremi;
\item non ha punti fissi.
\end{enumeratea}
\end{esercizio}

\begin{esercizio}
\label{ese:8.9}
Sono assegnati i punti $A(-5;0)$, $B(0;5)$ e $C(1;-1)$; determinate le coordinate dei vertici $A'B'C'$ del triangolo immagine di $ABC$ nella simmetria avente come centro il punto medio $M$ del lato $AC$.
\end{esercizio}

\begin{esercizio}
\label{ese:8.10}
I punti $A(1;5)$, $B(-2;2)$ e $C(0;-4)$ sono tre vertici di un parallelogramma. Determinate le coordinate del quarto vertice. Indicate con $M$ il punto di incontro delle diagonali; in $S_M$ il parallelogramma $ABCD$ è fisso o unito? Perché?
\end{esercizio}

\begin{esercizio}
\label{ese:8.11}
Sappiamo che l'equazione di una simmetria centrale di centro $C(p;q)$ è $\begin{cases}x'=2p-x\\y'=2q-y\end{cases}$; note le coordinate di un punto $P(x;y)$ e della sua immagine $P'(x';y')$ le coordinate del centro sono:
\begin{enumeratea}
\item $p=x'+x$ $q=y'+y$;
\item $p=x-\frac{1}{2}x'$ $q=y-\frac{1}{2}y'$;
\item $p=2(x'+x)$ $q=2(y'+y)$;
\item $p=\frac{1}{2}(x'+x)$ $q=\frac{1}{2}(y'+y)$;
\item $p=\frac{1}{2}(x'-x)$ $q=\frac{1}{1}(y'-y)$.
\end{enumeratea}
\end{esercizio}

\begin{esercizio}
\label{ese:8.12}
Verificate che i tre punti $A(3;2)$, $B(7;-2)$, $C(5;0)$ sono allineati. \`E vero che $C$ è il centro della simmetria che fa corrispondere al punto $A$ il punto $B$? (Ricorda che puoi verificare l'allineamento verificando che $\overline{AB}=\overline{AC}+\overline{CB}$)
\end{esercizio}

\begin{esercizio}
\label{ese:8.13}
Il centro della simmetria che associa al triangolo di vertici $A(0;4)$, $B(-2;1)$ e $C(1;5)$ il triangolo di vertici $A'(2;-2)$, $B'(4;1)$ e $C'(1;-3)$ è
\begin{multicols}{2}
\begin{enumeratea}
\item $K(-1;1)$;
\item $K(1;-1)$;
\item $K(1;1)$;
\item $K(-1;-1)$.
\end{enumeratea}
\end{multicols}
\end{esercizio}

\begin{esercizio}
\label{ese:8.14}
Determinate l'immagine $M'$ del punto medio $M$ del segmento $AB$ di estremi $A(0;5)$ e $B(-4;1)$ in $S_O$ ($O$ è l'origine del riferimento cartesiano). \`E vero che $BM'A$ è isoscele sulla base $AB$?
\end{esercizio}

\begin{esercizio}
\label{ese:8.15}
Determinate la natura del quadrilatero $ABA'B$ che si ottiene congiungendo nell'ordine i punti $A(-1;1)$, $B(-4;-5)$, $A'$ e $B'$ rispettivamente simmetrici di $A$ e $B$ in $S_O$. Determinate la misura delle sue diagonali.
\end{esercizio}

\begin{esercizio}
\label{ese:8.16}
Nel piano sono assegnati i punti $T$ e $T'$ corrispondenti in una simmetria assiale. Come potete determinare l'asse di simmetria?
\end{esercizio}

\begin{esercizio}
\label{ese:8.17}
Nel piano è assegnata la retta $r$ e un suo punto $P$ e un punto $P'$ non appartenente ad $r$. Costruisci la retta $r'$ immagine di $r$ nella simmetria assiale che fa corrispondere al punto $P$ il punto $P'$.
\end{esercizio}

\begin{esercizio}
\label{ese:8.18}
Costruite l'immagine di ciascun triangolo $ABC$ della figura~... nella simmetria avente come asse la retta del lato $AC$.
\end{esercizio}

\begin{esercizio}
\label{ese:8.19}
Nel triangolo isoscele $ABC$ di base $BC$ considerate la retta $r$ passante per $A$ e perpendicolare a $BC$; costruite l'immagine di $ABC$ nella simmetria di asse $r$. Stabilite quale proposizione è vera:
\begin{enumeratea}
\item il triangolo è fisso nella simmetria considerata;
\item il triangolo è unito nella simmetria considerata.
\end{enumeratea}
\end{esercizio}

\begin{esercizio}
\label{ese:8.20}
Assegnato il quadrato $ABCD$, determinate la sua immagine nella simmetria avente come asse la retta della diagonale $AC$. Stabilite quale proposizione è vera:
\begin{enumeratea}
\item il quadrato è fisso nella simmetria considerata;
\item il quadrato è unito nella simmetria considerata.
\end{enumeratea}
\end{esercizio}

\begin{esercizio}
\label{ese:8.21}
Motivate la verità delle proposizioni\\
$p_1$: <<il quadrato possiede 4 assi di simmetria>>,\\
$p_2$: <<il triangolo equilatero possiede 3 assi di simmetria>>.
\end{esercizio}

\begin{esercizio}
\label{ese:8.22}
Dimostrate che la retta di un diametro è asse di simmetria per la circonferenza. Potete concludere che la circonferenza possiede infiniti assi di simmetria?
\end{esercizio}

\begin{esercizio}
\label{ese:8.23}
Tra i trapezi ne trovate uno avente un asse di simmetria? Qual è l'asse di simmetria? 
\end{esercizio}

\begin{esercizio}
\label{ese:8.24}
Quali lettere dell'alfabeto, tra quelle proposte, hanno un asse di simmetria? (A, B, C, D, E, F)
\end{esercizio}

\begin{esercizio}
\label{ese:8.25}
Perché la retta che congiunge i punti medi dei lati obliqui di un trapezio isoscele non è un suo asse di simmetria?
\end{esercizio}

\begin{esercizio}
\label{ese:8.26}
Le due rette tracciate sono assi di simmetria del rettangolo $ABCD$ e pertanto lo sono anche per l'immagine in esso contenuta. Vero o falso?
\end{esercizio}

\begin{esercizio}
\label{ese:8.27}
In $S_x$ il segmento $AB$ di estremi $A(3;2)$ e $B(3;-2)$
\begin{enumeratea}
\item è unito, luogo di punti uniti;
\item non ha punti fissi;
\item ha tutti i suoi punti uniti tranne $A$ e $B$;
\item ha un solo punto fisso;
\item ha solo $A$ e $B$ fissi.
\end{enumeratea}
\end{esercizio}

\begin{esercizio}
\label{ese:8.28}
Dimostrate che un qualunque segmento $MN$ di estremi $M(a;b)$ e $N(c;d)$ ha come corrispondente sia nella simmetria avente come asse l'asse $x$, sia nella simmetria avente come asse l'asse $y$, il segmento $M'N'$ tale che $MN\cong M'N'$.

\noindent Ipotesi: $M(a;b)$, $N(c;d)$, $S_x:(M\rightarrow M') \wedge (N\rightarrow N')$
\\
Tesi: $MN\cong M'N'$\vspace{5pt}\\
\emph{Dimostrazione.}\\
Determino $\overline{MN}=\ldots{}$\\
Trovo $M'(\ldots{};\ldots{})$ e $N'(\ldots{};\ldots{})$\\
Determino $\overline{M'N'}=\ldots{}$\\
Concludo: \ldots{}

\noindent Ipotesi: $M(a;b)$, $N(c;d)$, $S_y:(M\rightarrow M') \wedge (N\rightarrow N')$
\\
Tesi: $MN\cong M'N'$\vspace{5pt}\\
\emph{Dimostrazione.}\\
Determino $\overline{MN}=\ldots{}$\\
Trovo $M'(\ldots{};\ldots{})$ e $N'(\ldots{};\ldots{})$\\
Determino $\overline{M'N'}=\ldots{}$\\
Concludo: \ldots{}
\end{esercizio}

\begin{esercizio}
\label{ese:8.29} % 33
Il triangolo $ABC$ è isoscele; sapendo che $A(0;4)$, $B(-2;0)$ e che l'asse $x$ è il suo asse di simmetria, determinate il vertice $C$, il perimetro e l'area del triangolo.
\end{esercizio}

\begin{esercizio}
\label{ese:8.30} % 34
Il triangolo $ABC$ è isoscele; sapendo che $A(0;4)$, $B(-2;0)$ e che l'asse $y$ è il suo asse di simmetria, determinate il vertice $C$, il perimetro e l'area del triangolo.
\end{esercizio}

\begin{esercizio}
\label{ese:8.31} % 35
Considerate la funzione di proporzionalità quadratica $y)2x^2$; rappresentatela nel riferimento cartesiano e segnate i suoi punti $A$, $B$ e $C$, rispettivamente di ascissa $x_A=1$, $x_B=-\frac{1}{2}$ e $x_C=frac{1}{\sqrt{2}}$; trovate i corrispondenti $A'$, $B'$, $C'$ nella simmetria $S_y$ e verificate che appartengono alla funzione assegnata. Vi è un punto della curva rappresentata che risulta fisso in $S_y$? Inoltre, quale delle seguenti affermazioni ritenete corretta:
\begin{enumeratea}
\item la curva è fissa nella simmetria considerata;
\item la curva è unita nella simmetria considerata.
\end{enumeratea}
\end{esercizio}

\begin{esercizio}
\label{ese:8.32} % 38
I punti $A(-5;1)$, $B(-2;6)$, $C(3;6)$ e $D(0;1)$ sono vertici di un quadrilatero.
\begin{enumeratea}
\item Dimostrate che è un parallelogrammo.
\item Determinate perimetro e area;
\item Determinate la sua immagine $A'B'C'D'$ in $S_{y=3}$.
\end{enumeratea}
\`E vero che sia sul lato $AB$ che sul lato $CD$ esiste un punto fisso nella simmetria considerata? Tali punti su quali lati di $A'B'C'D'$ si trovano? Perché?
\end{esercizio}

\begin{esercizio}
\label{ese:8.33} % 40
Determinate l'immagine del quadrilatero $ABCD$ di vertici $A(0;0)$, $B(2;2)$, $C(5;3)$, $D(0;5)$ nella simmetria $S_{b1}$.
\end{esercizio}

\begin{esercizio}
\label{ese:8.34} % 41
Nella simmetria $S_{b1}$ la retta $y=-x$ è fissa o unita?
\end{esercizio}

\begin{esercizio}
\label{ese:8.35} % 42
Motivate la verità della seguente proposizione: <<nella simmetria $S_{b2}$ l'immagine dell'asse $x$ è l'asse $y$>>. Viene mantenuto l'orientamento dell'asse $x$?
Completate: $S_{b2}:(\text{asse }x)\rightarrow (\text{asse } \ldots{})$ e $(\text{asse }y)\rightarrow(\ldots\ldots{})$
Analogamente: $S_{b1}:(\text{asse }x)\rightarrow (\text{asse } \ldots{})$ e $(\text{asse }y)\rightarrow(\ldots\ldots{})$
\end{esercizio}

\begin{esercizio}
\label{ese:8.36} % 43
Dato il quadrilatero $ABCD$ di vertici $A(0;0)$, $B(3;1)$, $C(4;4)$ e $D(1;3)$ trovate il suo corrispondente in $S_{b1}$. Quale delle seguenti affermazioni ritenete corretta:
\begin{enumeratea}
\item il quadrilatero è fisso nella simmetria considerata;
\item il quadrilatero è unito nella simmetria considerata.
\end{enumeratea}
\end{esercizio}

\begin{esercizio}
\label{ese:8.37} % 44
Determinate il corrispondente del parallelogramma $ABCD$ di vertici $A(-5;1)$, $B(-2;6)$, $C(3;6)$, $D(0;1)$ in $S_{b1}$; perché $AA'$, $BB'$, $CC'$ e $DD'$ sono paralleli? Ricordando che il parallelogramma ha un centro di simmetria, determinate il centro di simmetria di $ABCD$ e verificate che in $S_{b1}$ esso ha come immagine il centro di simmetria di $A'B'C'D'$.
\end{esercizio}

\begin{esercizio}
\label{ese:8.38} % 45
Nel piano cartesiano sono assegnati i punti $A(0;3)$, $B(-2;0)$ e $C(-1;-3)$.
\begin{enumeratea}
\item Determinate i punti $A'$, $B'$ e $C'$ immagine in $S_{b2}$.
\item Calcolate l'area del quadrilatero $A'B'C'O$, essendo $O$ l'origine del riferimento.
\item Motivate la verità della proposizione: <<i segmenti $AB$ e $A'B'$ si incontrano in un punto $P$ della bisettrice del II\textsuperscript{o}-IV\textsuperscript{o} quadrante>>.
\item \`E vero che $AP'B$ è congruente a $PAB'$?
\end{enumeratea}
\end{esercizio}

\begin{esercizio}
\label{ese:8.39} % 46
Sono assegnate le simmetrie
\[S_1:\begin{cases}x'=-x\\y'=-y\end{cases};\quad S_2:\begin{cases}x'=y\\y'=x\end{cases};\quad
S_3:\begin{cases}x'=2-x\\y'=y\end{cases};\quad S_4:\begin{cases}x'=-x-1\\y'=3-y\end{cases}\]
Usando qualche punto scelto arbitrariamente riconosci ciascuna di esse e completa la tabella sottostante:
\begin{center}
\begin{tabular}{cccc}
\toprule
Simmetria & Tipo & Centro (coordinate) & Asse (equazione)\\
\midrule
$S_1$ & & & \\
$S_2$ & & & \\
$S_3$ & & & \\
$S_4$ & & & \\
\bottomrule
\end{tabular}
\end{center}
\end{esercizio}

\begin{esercizio}
\label{ese:8.40} % 47
Quale tra le seguenti caratteristiche è invariante in una simmetria assiale?
\begin{enumeratea}
\item la posizione della figura;
\item la direzione della retta;
\item il parallelismo;
\item l'orientamento dei punti;
\item dipende dall'asse di simmetria.
\end{enumeratea}
\end{esercizio}

\begin{esercizio}
\label{ese:8.41} % 48
I segmenti $AB$ e $A'B'$ si corrispondono nella simmetria di asse $r$; sapendo che $ABB'A'$ è un rettangolo, quale proposizione è vera?
\begin{enumeratea}
\item $AB$ è perpendicolare ad $r$;
\item $AB$ è parallelo ad $r$;
\item $AB$ appartiene ad $r$;
\item $AB$ è obliquo rispetto ad $r$ e $AB\cap r=H$.
\end{enumeratea}
\end{esercizio}

\begin{esercizio}
\label{ese:8.42} % 49
\`E assegnato il punto $P\left(-\sqrt{3};\dfrac{\sqrt{2}-1}{2}\right)$. Determinate il suo corrispondente nelle simmetrie indicate e completate:
\[S_{b2}:P\rightarrow P'(\ldots{};\ldots{})\quad S_{x=-\dfrac{1}{2}}:P\rightarrow P'(\ldots{};\ldots{}) \quad S_{O}:P\rightarrow P'(\ldots{};\ldots{})\]
\[S_{x}:P\rightarrow P'(\ldots{};\ldots{})\quad S_{y=2}:P\rightarrow P'(\ldots{};\ldots{}) \quad S_{C(1;1)}:P\rightarrow P'(\ldots{};\ldots{})\]
\end{esercizio}

\begin{esercizio}
\label{ese:8.43} % 50
Un segmento unito in $S_{b2}$ è
\begin{enumeratea}
\item un segmento perpendicolare alla bisettrice del I\textsuperscript{o}-III\textsuperscript{o} quadrante;
\item un segmento perpendicolare alla bisettrice del II\textsuperscript{o}-IV\textsuperscript{o} quadrante nel suo punto medio;
\item un segmento parallelo alla bisettrice del I\textsuperscript{o}-III\textsuperscript{o} quadrante;
\item un segmento perpendicolare alla bisettrice del II\textsuperscript{o}-IV\textsuperscript{o} quadrante;
\item un segmento avente il suo punto medio appartenente alla bisettrice del II\textsuperscript{o}-IV\textsuperscript{o} quadrante.
\end{enumeratea}
\end{esercizio}

\begin{esercizio}
\label{ese:8.44} % 51
Nel piano sono assegnati i tre punti $A$, $B$ e $A'$; il punto $A'$ è immagine di $A$ in una traslazione; dopo aver determinato il vettore della traslazione costruite l'immagine del triangolo $ABA'$ (figura~...).
\end{esercizio}

\begin{esercizio}
\label{ese:8.45} % 52
Determinate l'immagine del parallelogrammo $ABCD$ nella traslazione di vettore $\vec{v} \equiv \overrightarrow{AC}$.
\end{esercizio}

\begin{esercizio}
\label{ese:8.46} % 53
Dati due punti distinti $A$ e $B$ e il vettore $\overrightarrow{CD}$ della figura~..., detti $A'$ e $B'$ i punti immagine di $A$ e $B$ nella traslazione di vettore $\overrightarrow{CD}$, rispondete alle domande:
\begin{enumeratea}
\item di che natura è il quadrilatero $ABB'A'$?
\item può succedere che il quadrilatero in questione sia un rettangolo? E un rombo?
\item cosa succede se $AB$ è parallelo al vettore $\overrightarrow{CD}$?
\end{enumeratea}
\end{esercizio}

\begin{esercizio}
\label{ese:8.47} % 54
Come dobbiamo assegnare due segmenti $AB$ e $A'B'$ affinché siano corrispondenti in una traslazione? \`E unica la traslazione che associa ad $AB$ il segmento $A'B'$?
\end{esercizio}

\begin{esercizio}
\label{ese:8.48} % 58
Nel riferimento cartesiano è assegnato il punto $P(-4;2)$. Determinate il punto $P'$ immagine nella traslazione $T(3;-1):\begin{cases}x'=x+3\\y'=y+(-1)\end{cases}$.\\
Strategia risolutiva:
\begin{enumerate*}
\item individuate il vettore $\vec{w}$ della traslazione: $\vec{w}(\ldots{};\ldots{})$;
\item tracciate il vettore nel riferimento cartesiano;
\item determinate le coordinate di $P'$: $P'(\ldots{};\ldots{})$.
\end{enumerate*}
Completate: $\overrightarrow{PP'}$ è \ldots\ldots\ldots{} a $\vec{w}$; questo significa che i due vettori hanno \ldots\ldots\ldots{} direzione (cioè sono \ldots\ldots\ldots{}), stesso \ldots\ldots\ldots{} e \ldots\ldots\ldots{} intensità.
\end{esercizio}

\begin{esercizio}
\label{ese:8.49} % 59
Nel riferimento cartesiano, dopo aver fissato il punto $P(-4;2)$ siano dati i punti $Q(\ldots{};\ldots{})$ e $Q'(\ldots{};\ldots{})$ immagine nella traslazione $T(3;-1)$. Dimostrate con le conoscenze di geometria sintetica che $PP'Q'Q$ è un parallelogramma.\\
\noindent Ipotesi: $PP'\cong QQ'$, $PP'\ldots{}QQ'$\\
Tesi: \ldots\ldots\ldots{}\\
\emph{Dimostrazione. \ldots\ldots{}}
\end{esercizio}

\begin{esercizio}
\label{ese:8.50} % 60
Sappiamo che l'equazione di una traslazione è $T(a;b):\begin{cases}x'=x+a\\y'=y+b\end{cases}$. Assegnate le coordinate $(x;y)$ di un punto $P$ e $(x';y')$ della sua immagine $P'$, le componenti del vettore della traslazione sono date da:
\begin{enumeratea}
\item $a=x'+x$\quad e\quad $b=y'+y$;
\item $a=x-x'$\quad e\quad $b=y-y'$;
\item $a=x'-x$\quad e\quad $b=y'-y$;
\item $a=x'+x$\quad e\quad $b=y'-y$;
\item $a=\dfrac{x'}{x}$\quad e\quad $b=\dfrac{y'}{y}$.
\end{enumeratea}
\end{esercizio}


\begin{comment}


\begin{esercizio}
\label{ese:8.???}
 38
I punti $A(-5;1)$, $B(-2;6)$, $C(3;6)$ e $D(0;1)$ sono vertici di un quadrilatero.
 1. Dimostrate che è un parallelogrammo
 2. Determinate perimetro e area
 3. Determinate la sua immagine A’B’C’D’  in 
 È vero che sia sul lato AB che sul lato CD esiste un punto fisso nella simmetria considerata? Tali punti su quali lati di A’B’C’D’ si trovano? Perché?
\end{esercizio}
 
\begin{esercizio}
\label{ese:8.???}
Dai vertici $B$ e $C$ dell'ipotenusa di un triangolo rettangolo $ABC$ traccia le rette rispettivamente parallele ai cateti $AC$ e $AB$; sia $D$ il loro punto di intersezione. Dimostrare che $ABDC\doteq 2\cdot ABC$ e che $MNPQ\doteq 2\cdot ABC$ dove $MNPQ$ è il rettangolo avente un lato congruente all'ipotenusa $BC$ e l'altro lato congruente all'altezza $AH$ relativa all'ipotenusa.
\end{esercizio}

\end{comment}
%\end{multicols}
