% (c) 2014 Daniele Masini - d.masini.it@gmail.com
\section{Esercizi}

\subsubsection*{\thechapter.2 - Poligoni equivalenti}

\begin{multicols}{2}

\begin{esercizio}
\label{ese:7.1}
Enunciate e dimostrate il teorema le cui ipotesi e tesi sono indicate di seguito.

\noindent Ipotesi: $AB\parallel DC$, $GH\perp AB$, $CJ\perp AB$, $AE\cong DE$, $CF\cong FB$.\\
Tesi: $ABCD\doteq GHJI$.
\end{esercizio}
 
\begin{esercizio}
\label{ese:7.2}
Dai vertici $B$ e $C$ dell'ipotenusa di un triangolo rettangolo $ABC$ traccia le rette rispettivamente parallele ai cateti $AC$ e $AB$; sia $D$ il loro punto di intersezione. Dimostrare che $ABDC\doteq 2\cdot ABC$ e che $MNPQ\doteq 2\cdot ABC$ dove $MNPQ$ è il rettangolo avente un lato congruente all'ipotenusa $BC$ e l'altro lato congruente all'altezza $AH$ relativa all'ipotenusa.
\end{esercizio}

\begin{esercizio}
\label{ese:7.3}
Costruire un rettangolo equivalente ad un trapezio dato.
\end{esercizio}

\begin{esercizio}
\label{ese:7.4}
Dimostrare che la mediana relativa ad un lato di un triangolo divide il triangolo dato in due triangoli equivalenti.
\end{esercizio}

\begin{esercizio}
\label{ese:7.5}
Dimostrare che in un parallelogramma $ABCD$ sono equivalenti i quattro triangoli determinati dalle diagonali $AC$ e $BD$.
\end{esercizio}

\begin{esercizio}
\label{ese:7.6}
Assegnato il trapezio $ABCD$, detto $E$ il punto di intersezione delle diagonali $DB$ e $AC$, dimostrare che $DEA$ è equivalente a $BEC$.
\end{esercizio}

\begin{esercizio}
\label{ese:7.7}
Dimostra che le diagonali di un trapezio lo dividono in quattro triangoli due dei quali sono equiestesi.
\end{esercizio}

\begin{esercizio}
\label{ese:7.8}
Dimostra che due triangoli sono equiestesi se hanno due lati ordinatamente congruenti e gli angoli tra essi compresi supplementari.
\end{esercizio}

\begin{esercizio}
\label{ese:7.9}
Dimostra che un triangolo $ABC$ è diviso da una sua mediana in due triangoli equiestesi.
\end{esercizio}

\end{multicols}

\subsection{Esercizi di applicazione dell'algebra alla geometria}

\begin{multicols}{2}

\begin{esercizio}
\label{ese:7.10}
Sia $ABC$ un triangolo con $AB=7$~cm, $BC=5$~cm, $AC=3$~cm. Condurre una parallela ad $AC$ che intersechi $AB$ in $D$ e $BC$ in $E$. Sapendo che $CE=BD$, trovare il perimetro del triangolo $BDE$. 	[$25/4$~cm]
\end{esercizio}

\begin{esercizio}
\label{ese:7.11}
Nel trapezio $ABCD$, le basi misurano 5~cm e 15~cm e l'area vale 120~cm\textsuperscript{2}. Determina la distanza tra la base maggiore ed il punto di intersezione dei lati obliqui del trapezio. 		[18~cm]
\end{esercizio}

\begin{esercizio}
\label{ese:7.12}
Sia $ABC$ un triangolo rettangolo in $A$, con $AB=8a$. Da un punto $D$ di $AC$ si tracci la parallela ad $AB$ che incontri $BC$ in $E$; sia $DE=6a$. Sapendo che $CDE$ e $ABED$ sono isoperimetrici, trovare l'area di $ABC$.   [$24a^2$]
\end{esercizio}

\begin{esercizio}
\label{ese:7.13}
Nel trapezio rettangolo $ABCD$ circoscritto ad una circonferenza  la base maggiore è $4/3$ dell'altezza ed il perimetro misura 48~cm. Trovare l'area del trapezio.              		[135cm2]
\end{esercizio}

\begin{esercizio}
\label{ese:7.14}
Sia $ABC$ un triangolo rettangolo con il cateto $AC = 32a$. Sapendo che $BC:AB=5:3$, trovare il perimetro del triangolo. Tracciare poi la parallela ad $AB$, che intersechi $CA$ in $D$ e $CB$ in $E$. Sapendo che $CD$ è medio proporzionale tra $CE$ ed $AB$, trovare l'area del trapezio $ABED$. 	[$2p=96a$; $A=93/2a^2$]
\end{esercizio}

\begin{esercizio}
\label{ese:7.15}
Sia $ABC$ un triangolo isoscele di base $BC=4$~cm e di area 40~cm\textsuperscript{2}. Dopo aver trovato la misura dell'altezza $AH$ si tracci l'altezza $CK$ e la si prolunghi di un segmento $KD$ tale che l'angolo $H\widehat{A}D$ sia congruente ad uno degli angoli alla base. Dopo aver dimostrato che $C\widehat{A}D$ è retto, trovare il perimetro del triangolo $CAD$. [$AH=20$~cm; $2p=90$~cm]
\end{esercizio}

\begin{esercizio}
\label{ese:7.16}
Due lati consecutivi di un parallelogramma misurano $2a$ e $4a$ e l'angolo tra essi compreso misura $60\grado$. Trovare la misura dell'area e delle diagonali. 	[$A=4a^2$; $d_1=2a$; $d_2=2a$]
\end{esercizio}

\begin{esercizio}
\label{ese:7.17}
Determinare perimetro ed area di un trapezio rettangolo circoscritto ad una circonferenza, sapendo che il lato obliquo è diviso dal punto di tangenza in due parti che misurano rispettivamente $4a$ e $9a$.  [$2p=50a$; $A=150a^2$]
\end{esercizio}

\begin{esercizio}
\label{ese:7.18}
Determinare perimetro ed area di un triangolo isoscele, sapendo che la base misura $10a$ e che l'angolo adiacente ad uno degli angoli alla base misura $150\grado$. 	[$2p=10a(2+\sqrt{3})$; $A=25\sqrt{3}a^2$].
\end{esercizio}

\begin{esercizio}
\label{ese:7.19}
Nel trapezio $ABCD$ la base maggiore, $AB$, misura 15~cm e la minore, $CD$, misura 5~cm. Prolungando i lati obliqui si ottiene un triangolo rettangolo. Trovare il perimetro del trapezio e del triangolo rettangolo $CDE$ sapendo che la differenza tra le due basi è uguale alla differenza tra il doppio di $BC$ e $AD$.	 [34~cm; 12~cm]
\end{esercizio}

\begin{esercizio}[Giochi di Archimede 2011]
\label{ese:7.20}
In un triangolo equilatero $ABC$ con lato di lunghezza 3~m, prendiamo i punti $D$, $E$ e $F$ sui lati $AC$, $AB$ e $BC$ rispettivamente, in modo che i segmenti $AD$ e $FC$ misurino 1~m e il segmento $DE$ sia perpendicolare ad $AC$. Quanto misura l'area del triangolo $DEF$? 	[3~m\textsuperscript{2}]
\end{esercizio}

\begin{esercizio}
\label{ese:7.21}
\`E dato un trapezio isoscele avente un angolo di $45\grado$ e il lato obliquo che misura 2~cm. Trovare l'area sapendo che la base minore misura $\sqrt{3}$~cm.  [$2+\sqrt{6}$~cm\textsuperscript{2}]
\end{esercizio}

\begin{esercizio}
\label{ese:7.22}
Nella circonferenza di diametro $BD$ sono inscritti i triangoli $ABD$ e $BDC$, con $A$ e $C$ da parti opposte rispetto a $BD$. Sia $H$ la proiezione di $C$ su $BD$. Sapendo che $AB=16$~cm e che il rapporto tra $AD$ e $BD$ e tra $BH$ e $HD$ è $3/5$, trovare il perimetro di $ABCD$. 	[$2p=28+5(\sqrt{6}+\sqrt{10})$~cm]
\end{esercizio}

\begin{esercizio}
\label{ese:7.23}
Dato il segmento $AB=1u$, costruite, spiegando ogni passaggio, il triangolo rettangolo $PQR$ i cui cateti $PQ$ e $PR$ misurano rispettivamente $\sqrt{2}u$ e $\sqrt{3}u$. Calcolate inoltre la misura dell'ipotenusa.
[$\sqrt{5}u$]
\end{esercizio}

\begin{esercizio}
\label{ese:7.24}
Il quadrato $ABCD$ ha il lato di 2~m; costruite sul lato $DC$ il triangolo isoscele $DEC$ di base $DC$ e avente $D\widehat{E}C=120\grado$; siano $F$ e $G$ i punti di intersezione delle rette $ED$ e $EC$ con la retta $AB$. Determinate la misura dei lati del triangolo $EFG$ spiegando brevemente i calcoli eseguiti.
\end{esercizio}

\begin{esercizio}
\label{ese:7.25}
\`E dato il triangolo equilatero $ABC$; la semiretta $r$ di origine $B$ è interna all'angolo $ABC$ e lo divide in due parti di cui $ABP=45\grado$, $P0r\cap AC$. Sapendo che la distanza di $P$ dal lato $AB$ è di 2~m, calcolate il perimetro del triangolo equilatero dato.
[$6+2\sqrt{3}$]
\end{esercizio}

\begin{esercizio}
\label{ese:7.26}
Su ciascun lato del triangolo equilatero $ABC$ costruite un quadrato. Sapendo che l'altezza del triangolo equilatero misura $3\sqrt{3}$~m, determinate il perimetro e l'area dell'esagono che si forma congiungendo i vertici dei quadrati. Costruite il rettangolo equivalente all'esagono.	[$18(1+\sqrt{3})$, $27(4+\sqrt{3})$]
\end{esercizio}

\begin{esercizio}
\label{ese:7.27}
Nel trapezio rettangolo $ABCD$ di base maggiore $AB$, l'angolo acuto di vertice $B$ misura $45\grado$ e l'altezza è di 8~m. Sapendo che la base minore è $3/4$ dell'altezza, determinate perimetro e area del trapezio.
[$28+8\sqrt{2}$, 80]
\end{esercizio}

\begin{esercizio}
\label{ese:7.28}
Nel parallelogramma $ABCD$ la diagonale minore $AC$ è perpendicolare al lato $BC$ e forma col lato $AB$ un angolo di $45\grado$. Sapendo che $AC=5$~m, calcolate il perimetro e l'area del parallelogramma.
[$10(1+\sqrt{2})$, $25/2$]
\end{esercizio}

\begin{esercizio}
\label{ese:7.29}
Il trapezio $ABCD$ di base maggiore $AB$, ha $\widehat{A}=45\grado$ e $\widehat{B}=60\grado$; sapendo che la base minore è uguale all'altezza che misura 12~cm, determinate perimetro e area del trapezio. [$24(9+\sqrt{3})$, $36+12\sqrt{2}+12\sqrt{3}$]
\end{esercizio}

\begin{esercizio}
\label{ese:7.30}
Il quadrilatero $ABCD$ è spezzato dalla diagonale $AC$ nel triangolo rettangolo isoscele $ABC$ retto in $B$ e nel triangolo $ADC$ isoscele su $AC$, avente l'altezza $DH$ doppia della base. Sapendo che $AB=5$~m, calcolate il perimetro e l'area del quadrilatero.
$\left[10+5\sqrt{34}\text{, }\frac{125}{2}\right]$
\end{esercizio}

\begin{esercizio}
\label{ese:7.31}
Il triangolo isoscele $ABC$ ha l'angolo in $A$ opposto alla base $BC$ di $120\grado$ ed è circoscritto ad una circonferenza di raggio $OH=\sqrt{6}$~m; calcolate perimetro e area del triangolo dato.  	
[$14\sqrt{2}+8\sqrt{6}$, $14\sqrt{3}+24$]
\end{esercizio}

\begin{esercizio}
\label{ese:7.32}
Nel triangolo $ABC$ l'angolo in $A$ misura $60\grado$ e sia $AE$ la sua bisettrice ($E$ su $BC$). Sapendo che $AE=8$~m, determinate la misura delle distanze $EH$ ed $EK$ del punto $E$ rispettivamente dai lati $AB$ e $AC$ e il perimetro del quadrilatero $AHEK$. \`E vero che tale quadrilatero è equivalente al triangolo equilatero di lato 8~m? \`E vero che tale quadrilatero può essere inscritto in una circonferenza? Se la risposta è affermativa stabilite il suo centro e determinate la misura di detta circonferenza e l'area del cerchio. 
[$\np{25,12}$; $\np{50,24}$]
\end{esercizio}

\begin{esercizio}
\label{ese:7.33}
Nel trapezio rettangolo $ABCD$ la base minore è metà dell'altezza. Determinate perimetro e area in funzione della misura $x$ della base minore nei casi in cui l'angolo acuto del trapezio è di
\begin{enumeratea}
\item $45\grado$;
\item $30\grado$;
\item $60\grado$.
\end{enumeratea}
\end{esercizio}

\begin{esercizio}
\label{ese:7.34}
Il triangolo $ABC$ è rettangolo e l'angolo di vertice $C$ misura $30\grado$; detta $AP$ la bisettrice dell'angolo retto, con $P$ su $BC$, e sapendo che $\overline{AP}=a$, determinate, in funzione di $a$, perimetro e area del triangolo dato. 
$\left[\frac{11}{6}a\sqrt{2}+a\sqrt{6}\text{; }\frac{1}{6}a^2(e+2\sqrt{3})\right]$
\end{esercizio}

\begin{esercizio}
\label{ese:7.35}
Il segmento $AC$ è la diagonale del quadrilatero $ABCD$ avente $A\widehat{B}C=C\widehat{A}D=90\grado$ e $B\widehat{C}A=A\widehat{D}C=60\grado$. \`E vero che $ABCD$ è un trapezio rettangolo? Calcolate perimetro e area del quadrilatero sapendo che $\overline{AC}=2a$.
$\left[a+3a\sqrt{3}\text{; }\frac{1}{2}a^2\sqrt{3}\right]$
\end{esercizio}

\begin{esercizio}
\label{ese:7.36}
Il quadrato $ABCD$ ha i suoi vertici sui lati del triangolo equilatero $HKL$ ($A$ e $B$ appartengono a $KL$, $C$ a $HL$ e $D$ a $HK$); sapendo che $\overline{AB}=3a$, calcolate il perimetro e l'area del triangolo equilatero.
$\left[6a\sqrt{3}+9a\text{; }\frac{1}{2}(21a^2\sqrt{3}+36a^2)\right]$
\end{esercizio}

\begin{esercizio}
\label{ese:7.37}
In un parallelogramma di area 12~m\textsuperscript{2}, le lunghezze di due lati consecutivi sono una il doppio dell'altra e uno degli angoli interni misura $60\grado$. Determina la lunghezza delle diagonali. [$2\sqrt[4]{27}$~m]
\end{esercizio}

\begin{esercizio}
\label{ese:7.38}
Nel triangolo $ABC$ di altezza $CH=8$~m, determina a quale distanza da $C$ si deve condurre una parallela al lato $AB$ in modo che il triangolo ottenuto sia equivalente alla metà di $ABC$. [$a\sqrt{2}$]
\end{esercizio}

\begin{esercizio}
\label{ese:7.39}
La base di un rettangolo è più lunga di 8~cm dell'altezza ed è più corta di 10~cm della diagonale. Calcola perimetro ed area del rettangolo. 			
\end{esercizio}

\begin{esercizio}
\label{ese:7.40}
In un triangolo equilatero $ABC$ di lato $l$ individua sul lato $AB$ un punto $P$ tale che detti $H$ e $K$ i piedi delle perpendicolari condotte da $P$ ai lati $AC$ e $BC$ risulti $\overline{PH}^2+\overline{PK}^2=\overline{PC}^2+\np{12,67}$
\end{esercizio}

\begin{esercizio}
\label{ese:7.41}
Un triangolo equilatero e un quadrato hanno lo stesso perimetro. Quanto vale il rapporto tra le aree delle due figure? 			[$16/9$]
\end{esercizio}

\begin{esercizio}
\label{ese:7.42}
In un triangolo rettangolo $ABC$, retto in $A$, si tracci una parallela $DE$ al cateto $AB$. Sapendo che l'area di $DEC$ è i $3/4$ di quella di $ABC$ e che $\overline{AC}$ misura 1~m, quanto misura $\overline{DC}$? [$3/4$]
\end{esercizio}

\begin{esercizio}
\label{ese:7.43}
Dato il quadrato $ABCD$ con $M$ punto medio di $AB$ ed $N$ punto medio di $CD$, tracciare i segmenti $AN$, $BN$, $DM$ e $CM$. Siano $P$ l'intersezione di $AN$ con $DM$ e $Q$ l'intersezione di $BN$ e $CM$. Che figura è $MQNP$? Quanti triangoli ci sono nella figura? Calcolare l'area di $MQNP$ e l'area di uno dei triangoli ottusangoli, sapendo che il lato del quadrato è 12~cm.
[36, 18]
\end{esercizio}

\begin{esercizio}
\label{ese:7.44}
Disegna un rombo con la diagonale minore lunga 6~cm e la diagonale maggiore 8~cm. Costruisci su ciascun lato del rombo un quadrato. Unisci i vertici liberi dei quadrati formando un ottagono. Calcolane l'area. Calcola anche l'area dei quattro triangoli che si sono formati. Calcola inoltre la misura degli angoli interni dell'ottagono.
[12~cm\textsuperscript{2}, 12~cm\textsuperscript{2}, $172\grado$]
\end{esercizio}

\begin{esercizio}
\label{ese:7.45}
Disegna un quadrato $ABCD$ e sul lato $AB$ poni i punti $M$ ed $N$ in modo che $AM\cong MN\cong NB$. Che figura è $MNCD$? Calcola il rapporto tra l'area di $MNCD$ e quella di $ABCD$. Calcola il perimetro di $MNCD$ sapendo che l'area del quadrato è 10~cm\textsuperscript{2}.
[\np{0,665}; \np{10,85}~cm]
\end{esercizio}

\begin{esercizio}
\label{ese:7.46}
Disegna un triangolo isoscele $ABC$ di base $AC=40$~mm e lato obliquo $AB=52$~mm. Costruisci sulla base $AC$ il triangolo $ACD$ di area doppia di $ABC$ e determina il perimetro del quadrilatero $ABCD$. Di che figura si tratta?
[\np{300,12}~mm]
\end{esercizio}

\begin{esercizio}
\label{ese:7.47}
Il parallelogramma $ABCD$ ha la base $AB$ lunga 12~cm e l'altezza di 6~cm. Disegna su $AB$ un punto $H$ e su $CD$ un punto $K$ tali che $DK=BH=3$~cm. Considera i due quadrilateri in cui il parallelogramma rimane diviso dal segmento $HK$: che quadrilateri sono? Calcolane l'area. Calcola inoltre il rapporto tra l'area di $HBCD$ e quella di $ABCD$.
[36; \np{0,625}]
\end{esercizio}

\begin{esercizio}
\label{ese:7.48}
Calcola l'altezza del rombo avente le diagonali di 36~cm e 48~cm. Calcola l'area del trapezio equivalente al rombo, sapendo che l'altezza del trapezio è di 24~cm e che la base maggiore è il doppio di quella minore.
\end{esercizio}

\begin{esercizio}
\label{ese:7.49}
Il rettangolo $R$ ha base $AB = 9$~cm e l'altezza $BC$ è i $4/3$ di $AB$. Calcola il perimetro e l'area di $R$. Disegna il parallelogramma $P$ equivalente al rettangolo $R$ e avente la base congruente alla diagonale del rettangolo. Calcola l'altezza di $P$. 
[42~cm; 108~cm\textsuperscript{2}; \np{7,2}~cm]
\end{esercizio}

\begin{esercizio}
\label{ese:7.50}
Calcola l'area del parallelogramma $P$ di base \np{4,5}~cm e altezza 2~cm e con il lato obliquo che è $5/4$ dell'altezza. Disegna la diagonale $AC$ e traccia l'altezza relativa ad $AB$ del triangolo $ABC$. Calcola l'area del triangolo $ABC$.
[\np{11,25}~cm\textsuperscript{2}; \np{5,625}~cm\textsuperscript{2}]
\end{esercizio}

\begin{esercizio}
\label{ese:7.51}
I lati del triangolo $ABC$ hanno le seguenti misure $AB=21$~cm, $BC=20$~cm e $AC=13$~cm; calcola l'area del parallelogramma $A'B'C'D'$ di base $AB\cong A'B'$, lato $AC\cong A'C'$ e diagonale $B'C'\cong BC$ (ricorda la formula di Erone).
[$252$~cm\textsuperscript{2}]
\end{esercizio}

\begin{esercizio}
\label{ese:7.52}
Dato il rombo $ABCD$, avente perimetro di 10~cm e la diagonale maggiore di 4~cm, calcola la misura della diagonale minore, l'area del rombo e la sua altezza. Considera un triangolo isoscele equivalente al rombo e avente la sua stessa altezza. Calcolane la misura di ciascun lato.
[3~cm; 6~cm\textsuperscript{2}; \np{2,4}~cm; 5~cm; \np{3,5}~cm]
\end{esercizio}

\begin{esercizio}
\label{ese:7.53}
Un rombo ha l'area di 336~cm\textsuperscript{2}, una diagonale uguale alla base di un triangolo di altezza \np{20,2}~cm e area di \np{141,4}~cm\textsuperscript{2}. Determina il perimetro del rombo.
[100~cm]
\end{esercizio}

\begin{esercizio}
\label{ese:7.54}
Determina l'area del quadrato formato dai 4 vertici liberi di 4 triangoli equilateri costruiti sui lati di un quadrato di lato 3~cm.
[\np{33,59}~cm\textsuperscript{2}]
\end{esercizio}

\begin{esercizio}
\label{ese:7.55}
Determina l'area del rombo intersezione di due triangoli equilateri costruiti sui lati opposti di un quadrato di lato 10~cm e aventi il vertice che cade internamente al quadrato.
[\np{15,48}~cm\textsuperscript{2}]
\end{esercizio}

\begin{esercizio}
\label{ese:7.56}
Determina le misure degli angoli del triangolo $AED$ formato disegnando le diagonali $EA$ e $AD$ di un esagono regolare $ABCDEF$.
\end{esercizio}

\begin{esercizio}
\label{ese:7.57}
Determina le misure degli angoli del triangolo $AEC$ formato disegnando le diagonali $EA$ ed $EC$ di un ottagono regolare $ABCDEFGH$.
\end{esercizio}

\begin{esercizio}
\label{ese:7.58}
Determina le misure degli angoli del triangolo $AFC$ formato disegnando le diagonali $AF$ e $FC$ di un ottagono regolare $ABCDEFGH$.
[$45\grado$; $67,5\grado$]
\end{esercizio}

\begin{esercizio}
\label{ese:7.59}
La differenza tra le diagonali di un rombo è 7~cm e una è $5/12$ dell'altra. Determina l'area di un triangolo isoscele il cui perimetro supera di 6~cm quello del rombo e la cui base è 8~cm. [\np{42,24}~cm\textsuperscript{2}]
\end{esercizio}

\begin{esercizio}
\label{ese:7.60}
Determinare l'area di un quadrilatero con le diagonali perpendicolari sapendo che l'una è $5/8$ dell'altra e che la loro somma è 39~cm.
[180~cm\textsuperscript{2}]
\end{esercizio}

\begin{esercizio}
\label{ese:7.61}
Determinare la misura degli angoli di un parallelogramma sapendo che uno degli angoli alla base è $2/7$ di quello adiacente.
\end{esercizio}

\begin{esercizio}
\label{ese:7.62}
In un quadrilatero un angolo è $93\grado8'42''$. Determinare l'ampiezza di ciascuno degli altri tre angoli sapendo che il secondo è $2/7$ del terzo e il terzo è $4/5$ del quarto. [$176\grado13'30''$; $50\grado21'$ ;$40\grado16'48''$]
\end{esercizio}

\begin{esercizio}
\label{ese:7.63}
Le dimensioni $a$ e $b$ di un rettangolo sono $a=\frac{3}{5}b$, il perimetro è 192~cm. Calcolane l'area.
[1080~cm\textsuperscript{2}]
\end{esercizio}

\begin{esercizio}
\label{ese:7.64}
In un rombo la differenza fra le diagonali è 8~cm e una diagonale è i $4/3$ dell'altra. Calcola area e perimetro del rombo.
[384~cm\textsuperscript{2}; 80~cm]
\end{esercizio}

\begin{esercizio}
\label{ese:7.65}
In un rombo la somma delle diagonali misura 196~cm, un quarto della misura della diagonale maggiore supera di 4~cm la misura della diagonale minore. Trova perimetro, area e altezza del rombo.
[328~cm; 2880~cm\textsuperscript{2}; \np{35,15}~cm]
\end{esercizio}

\begin{esercizio}
\label{ese:7.66}
In un trapezio rettangolo l'altezza è quadrupla della base minore e il lato obliquo è i $5/4$ dell'altezza. Determina l'area del trapezio sapendo che il suo perimetro è 70~cm.
[250~cm\textsuperscript{2}]
\end{esercizio}

\begin{esercizio}
\label{ese:7.67}
Il perimetro di un trapezio isoscele misura 124~cm e ciascun lato obliquo è lungo 30~cm. Determinane l'area e la misura della diagonale sapendo che una sua base è $7/25$ dell'altra.
[768~cm\textsuperscript{2}; 40~cm]
\end{esercizio}

\begin{esercizio}
\label{ese:7.68}
Determina l'area di un rettangolo sapendo che la misura della sua diagonale supera di 8~cm quella dell'altezza e che la differenza fra i $20/41$ della diagonale ed i $2/3$ dell'altezza è uguale ai $14/9$ della stessa altezza.
\end{esercizio}

\begin{esercizio}
\label{ese:7.69}
Il perimetro di un rettangolo misura 170~cm e l'altezza è $5/12$ della base. Trovare area e diagonale del rettangolo.
[\np{1500}~cm\textsuperscript{2}; 65~cm]
\end{esercizio}

\begin{esercizio}
\label{ese:7.70}
Il perimetro di un rettangolo misura 29~cm ed i $2/11$ della sua altezza sono uguali a $1/9$ della base. Trovare l'area del rettangolo.
[\np{49,5}~cm\textsuperscript{2}]
\end{esercizio}

\begin{esercizio}
\label{ese:7.71}
In un trapezio isoscele $ABCD$ avente la base maggiore $AB$, le diagonali sono fra loro perpendicolari e si intersecano in un punto $P$ che divide ogni diagonale in due parti con rapporto $5/12$. Calcola perimetro e area del trapezio, sapendo che la diagonale misura 68~cm.
[200~cm; \np{2304}~cm\textsuperscript{2}]
\end{esercizio}

\begin{esercizio}
\label{ese:7.72}
Un triangolo rettangolo ha ipotenusa 50~cm e un cateto 48~cm. Dal punto medio dell'ipotenusa tracciare la parallela al cateto minore. Determinare l'area di ciascuna delle due parti in cui è suddiviso il triangolo.
[84~cm\textsuperscript{2}; 252~cm\textsuperscript{2}]
\end{esercizio}

\begin{esercizio}
\label{ese:7.73}
In un triangolo l'altezza è 18~cm; se conducendo una parallela alla base, si divide il triangolo in due parti la cui superficie è in rapporto $16/25$, a quale distanza dal vertice è stata condotta la parallela?
\end{esercizio}

\begin{esercizio}
\label{ese:7.74}
Il triangolo $ABC$ ha base 14~cm e altezza 6~cm. Disegna la mediana $CM$ e calcola l'area dei triangoli $AMC$ e $MBC$. Come sono i triangoli?
\end{esercizio}

\begin{esercizio}
\label{ese:7.75}
La mediana di un triangolo è 12~cm. Determinare la misura di ciascuna delle parti in cui il baricentro divide la mediana.
\end{esercizio}

\begin{esercizio}
\label{ese:7.76}
Determinare la misura di una mediana $AM$ sapendo che $BM=8$~cm, dove $B$ è il baricentro del triangolo.
\end{esercizio}

\begin{esercizio}
\label{ese:7.77}
Determina la misura $BM$ del segmento appartenente alla mediana $AM$ in un triangolo equilatero $ABC$, avendo indicato con $B$ il baricentro. 
\end{esercizio}

\begin{esercizio}
\label{ese:7.78}
Determina il perimetro di un triangolo rettangolo sapendo che l'altezza relativa all'ipotenusa è 8~cm e che la proiezione di un cateto sull'ipotenusa è $4/3$ dell'altezza data.
[40~cm]
\end{esercizio}

\begin{esercizio}
\label{ese:7.79}
Determina la misura delle tre altezze del triangolo che ha i lati di 20~cm, 40~cm, 30~cm. (Suggerimento: Puoi ricorrere alla formula di Erone).
\end{esercizio}

\begin{esercizio}
\label{ese:7.80}
Il piede dell'altezza $CH$ di un triangolo $ABC$ divide la base $AB$ di 46~cm in due parti tali che $AH=\dfrac{9}{14}HB$; calcola l'area dei due triangoli $ACH$ e $BCH$, sapendo che $AC=24$~cm.
[$54\sqrt{7}$~cm\textsuperscript{2}; $84\sqrt{7}$~cm\textsuperscript{2}]
\end{esercizio}

\begin{esercizio}
\label{ese:7.81}
Trova il perimetro di un triangolo isoscele sapendo che la base è $2/3$ dell'altezza e che l'area è 24~cm\textsuperscript{2}.
\end{esercizio}

\begin{esercizio}
\label{ese:7.82}
Trova il perimetro di un triangolo isoscele sapendo che la base è $3/5$ dell'altezza e che l'area è 24~cm\textsuperscript{2}.
\end{esercizio}

\begin{esercizio}
\label{ese:7.83}
I lati del triangolo $ABC$ hanno le misure seguenti $AB=63$~cm, $BC=60$~cm e $AC=39$~cm; determina le misure delle tre relative altezze.
\end{esercizio}

\begin{esercizio}
\label{ese:7.84}
Determinare la misura di ciascun lato e l'area del triangolo isoscele avente il perimetro di 700~m, sapendo che la base e il lato obliquo sono in rapporto $\frac{16}{17}$.
[224~m; 238~m; \np{23520}~m\textsuperscript{2}]
\end{esercizio}

\begin{esercizio}
\label{ese:7.85}
Un trapezio rettangolo $ABCD$ è circoscritto ad una semicirconferenza con il centro $O$ sulla sua base maggiore $AB$ e raggio di misura 6~cm. Siano $S$ e $T$ i punti in cui tale semicirconferenza tange rispettivamente il lato obliquo $BC$ e la base minore $CD$. Sapendo che $AB$ misura 16~cm, calcolare le misure degli altri lati del trapezio. (Tracciare $OC$, $OS$, $OT$ e dimostrare che $OB$ è congruente a \ldots{}).
[6~cm; 10~cm; 8~cm]
\end{esercizio}

\begin{esercizio}
\label{ese:7.86}
Calcolare perimetro e area di un triangolo isoscele circoscritto a una semicirconferenza con il centro sulla sua base, sapendo che la base è i $3/2$ della relativa altezza e che il raggio della semicirconferenza misura 12~cm.
[80~cm, 300~cm\textsuperscript{2}]
\end{esercizio}

\begin{esercizio}
\label{ese:7.87}
Data una circonferenza di centro $O$, si consideri un punto $C$ esterno ad essa da cui si traccino le tangenti alla circonferenza stessa indicando con $A$ e $B$ i punti di tangenza. Sapendo che il segmento $AB$ misura 12~cm e che l'angolo $A\widehat{C}B$ ha ampiezza $60\grado$, calcolare il perimetro e l'area del quadrilatero $OACB$. Indicato poi con $E$ il punto in cui la retta $OB$ incontra la retta $AC$, calcolare il perimetro del triangolo $BCD$.
\end{esercizio}

\begin{esercizio}
\label{ese:7.88}
In un trapezio rettangolo, l'angolo che il lato obliquo forma con la base maggiore ha ampiezza $60\grado$ e la diagonale maggiore dimezza tale angolo; sapendo che la base minore misura 4~cm,  calcolare il perimetro del trapezio.
[$14 + 2\sqrt{3}$~cm]
\end{esercizio}

\begin{esercizio}
\label{ese:7.89}
In un rombo $ABCD$ ciascun lato misura 12~cm e l'angolo in $B$ ha ampiezza $120\grado$. Prendere sui lati $AB$, $BC$, $CD$ e $AD$ del rombo rispettivamente i punti $P$, $Q$, $S$ e $T$ in modo che i segmenti $AP$, $BQ$, $CS$ e $DT$ misurino 2~cm ciascuno. Calcolare il perimetro e l'area del quadrilatero $PQST$, dopo aver dimostrato che esso è un parallelogramma. (Tracciare da $T$ il segmento perpendicolare ad $AB$ e osservare i vari triangoli \ldots{}, analogamente tracciare poi da $P$ il segmento perpendicolare alla retta \ldots{}).  
\end{esercizio}

\begin{esercizio}
\label{ese:7.90}
Sul lato $AB$ di un triangolo equilatero $ABC$ avente area uguale a $25\sqrt{3}$~cm\textsuperscript{2}, si prenda il punto $P$ in modo che $AP$ misuri 4~cm; si tracci il segmento $PQ$ parallelo a $BC$ (con $Q$ appartenente ad $AC$) e lo si prolunghi di un segmento $QE$ congruente a $PQ$. Dopo aver dimostrato che il triangolo $APE$ è rettangolo, calcolare perimetro ed area del quadrilatero $CEPH$, essendo $H$ il piede dell'altezza del triangolo $ABC$ relativa ad $AB$.
[$9 + 5\sqrt{3} + 2\sqrt{7}$~cm; $29\sqrt{3}/2$~cm\textsuperscript{2}]
\end{esercizio}

\begin{esercizio}
\label{ese:7.91}
Data una semicirconferenza di centro $O$ e diametro $AB$ di misura $2r$, si tracci la corda $AC$ che forma con $AB$ un angolo di $30\grado$; si tracci quindi la tangente in $C$ alla semicirconferenza indicando con $D$ il punto in cui tale tangente incontra la retta $AB$ e con $E$ la proiezione ortogonale di $B$ sulla tangente stessa. Calcolare le misure dei segmenti $BC$, $CD$, $BE$, $CE$, $AE$. (Tracciare anche $CO$ \ldots{} osservare i vari angoli; per calcolare la misura di $AE$ tracciare la distanza di \ldots{} dalla retta \ldots{}).
\end{esercizio}

\begin{esercizio}
\label{ese:7.92}
Determina area e perimetro del quadrilatero $ABCD$ di coordinate $A(-1;7)$, $B(6;9/2)$, $C(4;-3)$ e $D(-4;3)$.
[\np{30,2}; \np{53,75}]
\end{esercizio}

\begin{esercizio}
\label{ese:7.93}
Determina area a perimetro del quadrilatero $ABCD$ di coordinate $A(0;3)$, $B(3;6)$, $C(6;3)$ e $D(-4;3)$. Che quadrilatero è? [\np{22,4}; \np{19,5}]
\end{esercizio}

\begin{esercizio}
\label{ese:7.94}
Determina l'area del quadrilatero $ABCD$ di coordinate $A(-8;5)$, $B(-2;11)$, $C(2;12)$ e $D(4;3)$.
[$A=14$]
\end{esercizio}

\begin{esercizio}
\label{ese:7.95}
Determina il quarto vertice $D$ del trapezio $ABCD$ di area $9$, sapendo che $A(-1;2)$, $B(5;2)$ e $C(3;4)$.
\end{esercizio}

\begin{esercizio}
\label{ese:7.96}
Determina il quarto vertice $D$ del parallelogramma $ABCD$ con $A(-3;-1)$, $B(4;1)$ e $C(3;4)$.
[$D(-4;2)$]
\end{esercizio}

\begin{esercizio}
\label{ese:7.97}
Verifica che il trapezio di vertici $A(-1;-1)$, $B(3;-2)$, $C\left(3;\frac{1}{2}\right)$ e $D\left(0;\frac{5}{2}\right)$ non è rettangolo. Calcola l'intersezione $E$ dei prolungamenti dei lati obliqui $BC$ e $AD$. Calcola inoltre il rapporto tra le aree dei triangoli $ABE$ e $CDE$.
\end{esercizio}

\begin{esercizio}
\label{ese:7.98}
Verifica che il quadrilatero di vertici $A(-2;-3)$, $B(3;-2)$, $C(4;1)$ e $D(0;3)$ è un trapezio e calcolane l'altezza.
\end{esercizio}

\begin{esercizio}
\label{ese:7.99}
Verifica che il quadrilatero di vertici $A(-4;1)$, $B(5;-2)$, $C(3;2)$ e $D(0;3)$ è un trapezio isoscele. Calcola l'intersezione $E$ dei prolungamenti dei lati obliqui $BC$ e $AD$. Calcola inoltre il rapporto tra le aree dei triangoli $ABE$ e $CDE$.
\end{esercizio}

\begin{esercizio}[Giochi di Archimede 2011]
\label{ese:7.100}
Nel quadrilatero $ABCD$ le diagonali sono ortogonali tra loro e gli angoli in $B$ e in $D$ sono retti. Inoltre $AB=AD=20$~cm e $BC=CD=30$~cm. Calcolare il raggio della circonferenza inscritta in $ABCD$.
[12~cm]
\end{esercizio}

\begin{esercizio}[Giochi di Archimede 2003]
\label{ese:7.101}
Sia dato un quadrato $ABCD$ di lato unitario e sia $P$ un punto interno ad esso tale che l'angolo $A\widehat{P}B$ misuri $75\grado$. Quanto vale la somma delle aree dei triangoli $ABP$ e $CDP$? 
\end{esercizio}

\begin{esercizio}[Giochi di Archimede 2003]
\label{ese:7.102}
Un parallelogramma di lati 1 e 2 ha un angolo di $60\grado$. Quanto misura la sua diagonale minore?   
\end{esercizio}

\begin{esercizio}[Giochi di Archimede 2007]
\label{ese:7.103}
In un triangolo $ABC$ scegliamo un punto $D$ su $AB$ e un punto $E$ su $AC$ in modo che la lunghezza di $AD$ sia un terzo di quella di $AB$ e la lunghezza di $AE$ sia un terzo di quella di $AC$. Sapendo che l'area del triangolo $ADE$ è 5~m\textsuperscript{2}, determinare l'area del quadrilatero $BCED$.
\end{esercizio}

\begin{esercizio}[Giochi di Archimede 2007]
\label{ese:7.104}
Il quadrato $ABCD$ nella figura~... ha il lato lungo 3~m. Il segmento $EF$ è lungo 1~m ed è parallelo ad $AB$. Quanto vale l'area dell'esagono $ABFCDE$?
\end{esercizio}

\begin{esercizio}[Giochi di Archimede 2007]
\label{ese:7.105}
Nella figura~... $ABCD$ è un quadrato avente la diagonale lunga 2~cm e $AEC$ è equilatero. Quanto vale l'area del quadrilatero $AECB$?
\end{esercizio}

\begin{esercizio}[Giochi d'Autunno 2010]
\label{ese:7.106}
Da un quadrato di lato 10~cm si tagliano i quattro angoli in modo da ottenere un ottagono regolare (figura~...). Quanto è lungo il lato dell'ottagono?
\end{esercizio}

\begin{esercizio}[Giochi di Archimede 2006]
\label{ese:7.107}
Nella figura~..., il segmento $DE$ è parallelo ad $AB$. Sapendo che l'area di $DEC$ è uguale ai $3/4$ di quella di $ABC$ e che $AC$ misura 1~m, quanto misura $DC$?
\end{esercizio}

\begin{esercizio}[Giochi di Archimede 2005]
\label{ese:7.108}
Il triangolo $ABC$ è rettangolo ed i cateti $AB$ e $AC$ misurano rispettivamente 3~m e 4~m. Siano $B'$ e $C'$ punti appartenenti rispettivamente ai lati $AB$ e $AC$, tali che la retta contenente il segmento $B'C'$ sia parallela a quella contenete il segmento $BC$ e distante 1~m da essa (figura~...), Calcolare l'area del triangolo $AB'C'$.
\end{esercizio}

\begin{esercizio}[Giochi d'Autunno 2011]
\label{ese:7.109}
L'area di un bosco è rappresentata dalla figura~... di vertici $F$, $O$, $I$, $N$ che è un parallelogramma la cui base misura \np{1001}~m e la cui altezza misura \np{2012}~m. Il punto $S$ si trova sulla base $NI$ a 143~m dal vertice $I$. Qual è l'area del quadrilatero $BOIS$?
\end{esercizio}

\begin{esercizio}[Giochi d'Autunno 2011]
\label{ese:7.110}
Nel parallelogramma $ABCD$ della figura~... il segmento $BD$ è perpendicolare ad $AB$ ed $E$ e $F$ sono i punti medi di $AB$ e $CD$ rispettivamente. Calcolare l'area del quadrilatero $GEHF$, sapendo che $AB=5$~cm e $BD=2$~cm.
\end{esercizio}

\begin{esercizio}[Giochi d'Autunno 2010]
\label{ese:7.111}
In un triangolo due angoli misurano rispettivamente $30\grado$ e $105\grado$ ed il lato tra essi compreso è lungo 2~cm. Qual è la misura del perimetro del triangolo? 
\end{esercizio}

\begin{esercizio}[Giochi d'Autunno 2011]
\label{ese:7.112}
In un parallelogramma di area 1~m\textsuperscript{2} le lunghezze di due lati consecutivi sono una il doppio dell'altra. Inoltre uno degli angoli interni misura $60\grado$. Quanto misura la diagonale minore?
\end{esercizio}

\begin{esercizio}[Giochi d'Autunno 2010]
\label{ese:7.113}
In un triangolo equilatero $ABC$ con lato di lunghezza 3~m, prendiamo i punti $D$, $E$ e $F$ sui lati $AC$, $AB$ e $BC$ rispettivamente, in modo che i segmenti $AD$ e $FC$ misurino 1~m e il segmento $DE$ sia perpendicolare ad $AC$. Quanto misura l'area del triangolo $DEF$?
\end{esercizio}

\begin{esercizio}[Giochi di Archimede 2005]
\label{ese:7.114}
Dato un quadrato $ABCD$ si uniscono i punti medi dei lati aventi un vertice in comune formando un nuovo quadrato $EFGH$. Ripetiamo la stessa operazione per $EFGH$ e otteniamo un nuovo quadrato $A'B'C'D'$. Quanto vale il rapporto tra l'area di $ABCD$ e l'area di $A'B'C'D'$?
\end{esercizio}

\end{multicols}
